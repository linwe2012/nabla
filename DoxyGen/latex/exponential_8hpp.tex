\hypertarget{exponential_8hpp}{}\section{D\+:/\+V\+C\+Repos/\+Open\+G\+L\+T6/\+Open\+G\+L\+T6/third-\/party/include/glm/exponential.hpp File Reference}
\label{exponential_8hpp}\index{D:/VCRepos/OpenGLT6/OpenGLT6/third-\/party/include/glm/exponential.hpp@{D:/VCRepos/OpenGLT6/OpenGLT6/third-\/party/include/glm/exponential.hpp}}
{\ttfamily \#include \char`\"{}detail/func\+\_\+exponential.\+hpp\char`\"{}}\newline


\subsection{Detailed Description}
\mbox{\hyperlink{group__core}{G\+LM Core}} 